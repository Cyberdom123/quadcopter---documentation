\section{\dumblang{Realizacja projektu}{Project implementation}}
\dumblang{
Mikroprocesorem użytym do realizacji jest STM32F103C6T6, posiadający 32K pamięci ROM mikroprocesor ARM Cortex-M3. F103 posiada wbudowaną obsługę magistrali I$^2$C oraz SPI, używanych do komunikacji z innymi peryferiami na pokładzie quadcoptera.

Jako sensor użyty został układ MPU6050. Jest to jednostka inercjalna (IMU) - układ MEMS łączący akcelerometr oraz żyroskop. W celu łatwego i intuicyjnego odczytu danych z sensorów z poziomu kodu, została napisana biblioteka do inicjalizacji i obsługi MPU6050. Wykorzystuje ona wbudowany w mikroprocesor interfejs szeregowy I$^2$C, przez który odbywa się komunikacja między układami. Transmisja po magistrali odbywa się z częstotliwością 400kHz. W celu zaoszczędzenia czasu procesora wkyorzystano kontroller DMA, pozwalający na przesył danych z pamięci do układu peryferyjnego.

Analogiczna biblioteka została napisana do obługi układu radiowego nRF24L01. Układ nRF służy do cyfrowej komunikacji radiowej między quadcopterem a komputerem PC. Jest on podłączony do mikroprocesora poprzez magistralę SPI i również wykorzystuje układ DMA. Drugi układ nRF24L01 komunikujący się z dronem został podłączony do komputera osobistego. Z komputera można odczytać dane telemetryczne oraz sterować dronem poprzez podpięty do komputera kontroler Xboxa 360.

Aby utrzymać stabilność quadcoptera, konieczne jest utrzymanie go pod określonym (stabilnym) kątem podczas przemieszczania. Na przykład, gdy chcemy, aby dron wznosił się, kąty roll i pitch powinny być ustawione na 0 stopni. Układ IMU dostarcza informacji o przyspieszeniu i prędkości kątowej, a odpowiedni model przetwarzania tych danych umożliwia odczytanie kątów. Chociaż można odczytać kąty z akcelerometru przy użyciu funkcji trygonometrycznych, to samo zastosowanie akcelerometru do bezpośredniego wyznaczania kątów jest niewystarczające ze względu na niedokładności pomiarów i podatność na szumy. W celu określenia kątów, moglibyśmy również skorzystać z żyroskopu i całkować prędkości kątowe w ustalonych odstępach czasu. Jednak, podobnie jak w przypadku akcelerometru, samo to podejście nie jest wystarczające. Pomimo wysokiej dokładności, żyroskop jest podatny na dryft, czyli stopniowe odchylenie od rzeczywistej wartości. Prędkość kątowa ulega drobnym przesunięciom, co prowadzi do narastania błędu w wyniku procesu całkowania. Kąty uzyskane poprzez całkowanie będą poprawne w odniesieniu do układu związanego z obiektem. Jednakże, aby uzyskać kąty odniesione do ziemi, konieczne jest ich przekształcenie. W tym celu używana jest macierz rotacji Eulera, która pozwala dostosować kąty do układu odniesienia związanego z ziemią.  

Ostatecznie, integracja danych z obu czujników za pomocą algorytmów filtracji, takich jak filtr komplementarny lub filtr Kalmana, pozwala na uzyskanie precyzyjnych informacji o kątach i skuteczną stabilizację quadcoptera w trakcie lotu. W naszym projecie został użyty filtr kalmana, jest to algorytm mający na celu estymację rzeczywistego stanu systemu na podstawie zaobserwowanych pomiarów z szumem Gaussowskim. Algorytm ten działa w dwóch głównych krokach: przewidywaniu stanu systemu na podstawie poprzedniego stanu i uwzględnieniu zmiany za pomocą nowego pomiaru. Filtr Kalmana korzysta z matematycznego modelu systemu oraz informacji o szumie pomiarowym i procesowym, aby minimalizować błędy estymacji i dostarczyć dokładniejszą reprezentację rzeczywistego stanu systemu. W efekcie po wielu iteracjach dostarcza on optymalną estymatę stanu systemu, uwzględniając dynamiczne zmiany i niepewności pomiarów.

Estymaty kątów oraz prędkość kątowa yaw podawana jest na układ regulacji składający się z trzech regulatorów PID dla każdej z wielkości.
Regulator PID działa na zasadzie korekty błędów pomiędzy rzeczywistym stanem systemu a pożądanym stanem referencyjnym. Składa się z trzech składowych: proporcjonalnej (P), która reaguje proporcjonalnie do bieżącego błędu, całkującej (I), która kumuluje błędy w czasie, oraz różniczkującej (D), która uwzględnia tempo zmian błędu. Wyjściem regulatorów PID jest wypełnienie które docelowo podawane jest na mieszacz sygnałów, który w wyniku zadaje wypełnienie każdemu z silników.}
{
The MCU used in the project implementation is STM32F103C6T6, an ARM Cortex-M3 microcontroller, containing 32Kbytes of ROM. F1303 has built-in hardware support for I$^2$C and SPI busses used for data transmission between on-board peripherals.

The MPU6050 has been used as a sensor. It is an inertial measurement unit (IMU), a MEMS integrated circuit combining in itself an accelerometer and a gyroscope. For easy and intuitive reading of the sensor data at the code level, a library for initialising and handling the MPU has been written. It uses the built-in I$^2$C bus which is used to exchange data between the IMU and MCU. The transmission is made in 400kHz mode. In order to save MCU execution time, a DMA controller has been used, allowing for sending data from RAM to the peripheral.

Similarly, a library for nRF24L01 radio module has been written. nRF is being used for digital radio transmission between the UAV and a PC. It is connected to the MCU by the SPI bus and also uses the DMA mechanism. Another nRF connected to the PC is used to communicate with the quadcopter. The transceivers are used in order to read telemetry data and for piloting the UAV using an Xbox controller connected to the PC.

To keep the UAV stable it is essential to keep it at a set (stable) angle during movement. For instance, when we want the drone to fly up vertically, the roll and pitch angle have to be set to 0 degrees. The IMU supplies the data about displacement and angular velocity and a suitable mathematical model processes the data in order to estimate the angles. 

Finally the integration of data from both sensors with the help of filtration algorithms, such as complementary filter or Kalman filter allows us to gain precise information about the angles and for effective stabilisation of the quadcopter during flight. In our project we used the Kalman filter. It is an aimed at estimating the actual state of the system based on the observer measurements with Gaussian noise. The algorithm works in two main steps: estimation of the state of the system based on previous state and correcting it during the next measurement. Kalman filter uses the mathematical model and data about Gaussian noise in order to minimalise the estimates and measurement unceirtanties. 

The angle estimates and yaw angular velocity is fed into the regulation system consisitng of three PID regulators for every quantity. 
The PID regulator works by minimising the difference between the actual state of the system and a desirable reference state. It contains three components: proportional (P), which reacts proportionally to current deviateion, integral (I), which accumulates the deviation as time passes and differential (D), which takes into account the rate of change of the difference. The output of the PIDs is a duty cycle fed into a signal mixer, which sets the duty cicle for each of the motors.
}